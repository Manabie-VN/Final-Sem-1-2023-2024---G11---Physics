\chapter{Tóm tắt lý thuyết và công thức}
\section{Định luật vạn vật hấp dẫn}
Lực hấp dẫn giữa hai chất điểm tỉ lệ thuận với tích hai khối lượng và tỉ lệ nghịch với bình phương khoảng cách giữa chúng.
$$F_\text{hấp dẫn}=G\dfrac{m_Am_B}{r^2}$$
với:
\begin{itemize}
	\item $m_A$, $m_B$: khối lượng hai chất điểm, đơn vị trong hệ SI là kilogram $\left(\si{\kilogram}\right)$,
	\item $r$: khoảng cách giữa hai chất điểm, đơn vị trong hệ SI là mét $\left(\si{\meter}\right)$,
	\item $G$: hằng số hấp dẫn, có giá trị thực nghiệm $G=\SI{6.68E-11}{\dfrac{\newton\meter^2}{\kilogram^2}}$.
\end{itemize}
\section{Trường hấp dẫn}
\subsection{Định nghĩa}
Trường hấp dẫn là trường vật chất bao quanh một vật có khối lượng và là môi trường truyền tương tác giữa các vật có khối lượng. Tính chất cơ bản của trường hấp dẫn là tác dụng lực hấp dẫn lên vật có khối lượng khác đặt trong nó.
\subsection{Tính chất}
Trường hấp dẫn của một vật khối lượng $M$ có các tính chất:
\begin{itemize}
	\item Tương tác hấp dẫn là tương tác hút.
	\item Các đường sức của trường hấp dẫn luôn hướng vào tâm của vật sinh ra trường hấp dẫn.
	\item Phạm vi tác dụng của trường hấp dẫn rất lớn. Tuy nhiên, càng ra xa vật $M$ thì độ lớn của lực hấp dẫn càng giảm.
	\item Trường hấp dẫn chỉ được xem gần đúng là trường đều khi xét trong vùng không gian rất nhỏ.
\end{itemize}
\section{Cường độ trường hấp dẫn}
Cường độ trường hấp dẫn là đại lượng đặc trưng cho trường hấp dẫn về phương diện tác dụng lực lên các vật có khối lượng đặt trong trường hấp dẫn.
$$g=\dfrac{GM}{r^2}$$
với:
\begin{itemize}
	\item $G$: hằng số hấp dẫn, có giá trị thực nghiệm $G=\SI{6.68E-11}{\dfrac{\newton\meter^2}{\kilogram^2}}$,
	\item $M$: khối lượng vật thể gây ra trường hấp dẫn, đơn vị trong hệ SI là kilogram $\left(\si{\kilogram}\right)$,
	\item $r$: khoảng cách từ chất điểm $M$ đến điểm đang xét, đơn vị trong hệ SI là mét $\left(\si{\meter}\right)$.
\end{itemize}
Nếu xem Trái Đất dạng hình cầu đồng nhất khối lượng $M_\text{TĐ}$ và bán kính $R_\text{TĐ}$ thì cường độ trường hấp dẫn của một điểm trên mặt cầu này là
$$g=G\dfrac{M_\text{TĐ}}{\left(R_\text{TĐ}+h\right)^2}$$
Tại điểm ở gần mặt đất $\left(h=0\right)$ thì
$$g_0=G\dfrac{M_\text{TĐ}}{R^2_\text{TĐ}}\approx\SI{9.81}{\meter/\second^2}.$$
\section{Thế năng hấp dẫn - Thế hấp dẫn}
\subsection{Thế năng hấp dẫn}
Thế năng hấp dẫn là đại lượng đặc trưng cho năng lượng tương tác hấp dẫn giữa vật có khối lượng $M$ và vật có khối lượng $m$.\\
Thế năng hấp dẫn tại một điểm trong trường hấp dẫn do vật có khối lượng $M$ sinh ra là công cần thực hiện để dịch chuyển một vật có khối lượng $m$ từ điểm đó ra xa vô cùng
$$W=-G\dfrac{mM}{r}$$
với $r$ là khoảng cách giữa hai vật.
\subsection{Thế hấp dẫn}
Thế hấp dẫn tại một điểm trong trường hấp dẫn của một vật có khối lượng $M$ gây ra là đại lượng đặc trưng cho khả năng tạo ra thế năng hấp dẫn cho các vật khác đặt tại điểm đó
$$\Phi=-G\dfrac{M}{r}$$
với $r$ là khoảng cách từ $M$ đến điểm đang xét.\\
Thế năng hấp dẫn của vật có khối lượng $m$ đặt tại một điểm trong trường hấp dẫn được xác định
$$W=m\Phi.$$