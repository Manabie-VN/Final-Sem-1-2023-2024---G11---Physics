\chapter{Câu hỏi ôn tập}
\ANSMCQ{
	\begin{center}
		\begin{tabular}{|m{2.8em}|m{2.8em}|m{2.8em}|m{2.8em}|m{2.8em}|m{2.8em}|m{2.8em}|m{2.8em}|m{2.8em}|m{2.8em}|}
			\hline
			1C & 2A & 3B & 4A & 5D & 6C & 7D & 8D & 9A & 10A\\
			\hline
			11B & 12C & 13B & 14D & 15D &  &  &  &  &\\
			\hline
		\end{tabular}
\end{center}}
\begin{enumerate}[label=\bfseries Câu \arabic*:]
	\item Câu nào sau đây là đúng khi nói về lực hấp dẫn do Trái Đất tác dụng lên Mặt Trời và do Mặt Trời tác dụng lên Trái Đất?
	\begin{mcq}
		\item Hai lực này cùng phương, cùng chiều.
		\item Hai lực này cùng chiều, cùng độ lớn.
		\item Hai lực này cùng phương, ngược chiều, cùng độ lớn.
		\item Phương của hai lực này luôn thay đổi và không trùng nhau.
	\end{mcq}
\hideall{
\textbf{Đáp án C.}
}

\item Với $G$ là hằng số hấp dẫn, $M$ là khối lượng Trái Đất, $R$ là bán kính Trái Đất, $h$ là độ cao của vật nặng, $m$ là khối lượng vật nặng thì gia tốc rơi tự do của một vật ở gần mặt đất được tính bởi công thức
\begin{mcq}(4)
	\item $g=\dfrac{GM}{R^2}$.
	\item $g=\dfrac{GM}{\left(R+h\right)^2}$.
	\item $g=\dfrac{GMm}{R^2}$.
	\item $g=\dfrac{GMm}{\left(R+h\right)^2}$.
\end{mcq}
\hideall{
\textbf{Đáp án A.}
}

\item Đơn vị của hằng số hấp dẫn là
\begin{mcq}(4)
	\item $\si{\kilogram\meter/\second^2}$.
	\item $\si{\newton\meter^2/\kilogram^2}$.
	\item $\si{\meter/\second^2}$.
	\item $\si{\newton\meter/\second}$.
\end{mcq}
\hideall{
\textbf{Đáp án B.}
}

\item Hai tàu thuỷ, mỗi chiếc có khối lượng 50000 tấn ở cách nhau $\SI{1}{\kilo\meter}$. So sánh lực hấp dẫn giữa chúng với trọng lượng của một quả cân có khối lượng $\SI{20}{\gram}$. Lấy $g=\SI{10}{\meter/\second^2}$.
\begin{mcq}(2)
	\item Nhỏ hơn.
	\item Bằng nhau.
	\item Lớn hơn.
	\item Chưa đủ dữ kiện để xác định.
\end{mcq}
\hideall{
\textbf{Đáp án A.}\\
Lực hấp dẫn giữa hai tàu thuỷ:
$$F=\dfrac{Gm^2}{r^2}=\left(\SI{6.68E-11}{\newton\meter^2/\kilogram^2}\right)\cdot\dfrac{\left(\SI{5E7}{\kilogram}\right)^2}{\left(\SI{1000}{\meter}\right)^2}=\SI{0.167}{\newton}$$
Trọng lượng của vật nặng $\SI{20}{\gram}$:
$$P=mg=\left(\SI{20E-3}{\kilogram}\right)\cdot\left(\SI{10}{\meter/\second^2}\right)=\SI{0.2}{\newton}$$
Vậy lực hấp dẫn giữa hai tàu thuỷ nhỏ hơn trọng lượng của vật có khối lượng $\SI{20}{\gram}$.

}

\item Chọn phát biểu \textbf{sai}
\begin{mcq}
	\item Các hành tinh đều chuyển động trên các quỹ đạo elip với mặt trời là một tiêu điểm. 
	\item Coi quỹ đạo chuyển động của các hành tinh gần đúng là tròn thì lực hấp dẫn tác dụng lên hành tinh đã gây ra gia tốc hướng tâm.
	\item Tốc độ tối thiểu để đưa vệ tinh lên quỹ đạo tròn quanh Trái Đất là tốc độ vũ trụ cấp I.
	\item Nếu tốc độ đầu để đưa vệ tinh lên quỹ đạo lớn hơn tốc độ vũ trụ cấp I thì vệ tinh sẽ đi xa khỏi Trái Đất theo quỹ đao parabol.
\end{mcq}
\hideall{
\textbf{Đáp án D.}\\
Đáp án D sai vì nếu tốc độ đầu của vệ tinh lớn hơn tốc độ vũ trụ cấp I nhưng chưa đạt đến tốc độ vũ trụ cấp II thì vệ tinh chưa thoát ra khỏi trường hấp dẫn của Trái Đất, do đó nó sẽ chuyển động xung quanh Trái Đất.
}

\item Khi khối lượng của hai vật và khoảng cách giữa chúng đều giảm đi phân nửa thì lực hấp dẫn giữa chúng có độ lớn 
\begin{mcq}(2)
	\item giảm đi 8 lần.
	\item giảm đi một nửa.
	\item giữ nguyên như cũ.
	\item tăng gấp đôi.
\end{mcq}
\hideall{
\textbf{Đáp án C.}\\
$$\dfrac{F'}{F}=\dfrac{m'_1m'_2}{m_1m_2}\cdot\left(\dfrac{r}{r'}\right)^2=\dfrac{1}{4}\cdot2^2=1$$
}

\item Chỉ ra kết luận \textbf{sai} trong các kết luận sau đây
\begin{mcq}
	\item Trọng lực của một vật được xem gần đúng là lực hút của Trái Đất tác dụng lên vật đó.
	\item Trọng lực có chiều hướng về phía Trái Đất.
	\item Trọng lực của một vật giảm khi đưa vật lên cao hoặc đưa vật từ cực bắc trở về xích đạo.
	\item Trên Mặt Trăng, nhà du hành vũ trụ có thể nhảy lên rất cao so với khi nhảy ở Trái Đất vì ở đó khối lượng và trọng lượng của nhà du hành giảm.
\end{mcq}
\hideall{
\textbf{Đáp án D.}\\
Khi lên Mặt Trăng, gia tốc trọng trường giảm dẫn đến trọng lượng của nhà du hành giảm, khối lượng của nhà du hành không thay đổi.
}

\item {Một vật ở trên mặt đất có trọng lượng $\SI{9}{\newton}$. Khi ở một điểm cách tâm Trái Đất $3R$ ($R$ là bán kính Trái Đất) thì nó có trọng lượng là
\begin{mcq}(4)
	\item $\SI{81}{\newton}$.
	\item $\SI{27}{\newton}$.
	\item $\SI{3}{\newton}$.
	\item $\SI{1}{\newton}$.
\end{mcq}}
\hideall{
	\textbf{Đáp án D.}\\
	$$\dfrac{P'}{P}=\dfrac{g'}{g}=\left(\dfrac{r}{r'}\right)^2=\left(\dfrac{1}{3}\right)^2=\dfrac{1}{9}\Rightarrow P'=\dfrac{P}{9}=\SI{1}{\newton}.$$
}

\item Gia tốc rơi tự do ở bề mặt Mặt Trăng là $g_0$ và bán kính Mặt Trăng là $\SI{1740}{\kilo\meter}$. Ở độ cao $h=\SI{3480}{\kilo\meter}$ so với bề mặt Mặt Trăng thì gia tốc rơi tự do bằng
\begin{mcq}(4)
	\item $g_0/9$.
	\item $g_0/3$.
	\item $3g_0$.
	\item $9g_0$.
\end{mcq}
\hideall{
\textbf{Đáp án A.}\\
$$\dfrac{g}{g_0}=\left(\dfrac{R}{R+h}\right)^2=\left(\dfrac{\SI{1740}{\kilo\meter}}{\SI{1740}{\kilo\meter}+\SI{3480}{\kilo\meter}}\right)^2=\dfrac{1}{9}\Rightarrow g=\dfrac{g_0}{9}.$$
}

\item Biết bán kính Trái Đất là $R$. Lực hút của Trái Đất đặt vào một vật khi vật đó ở trên mặt đất là $\SI{45}{\newton}$. Khi lực hút do Trái Đất tác dụng lên vật là $\SI{5}{\newton}$ thì vật ở độ cao bằng
\begin{mcq}(4)
	\item $2R$.
	\item $9R$.
	\item $\dfrac{2R}{3}$.
	\item $\dfrac{R}{9}$.
\end{mcq}
\hideall{
\textbf{Đáp án A.}\\
$$\dfrac{F'}{F}=\left(\dfrac{R}{R+h}\right)^2\Leftrightarrow \dfrac{R}{R+h}=\dfrac{1}{3}\Rightarrow h=2R.$$
}

\item Hãy tính gia tốc rơi tự do trên bề mặt của Mộc Tinh. Biết gia tốc rơi tự do trên bề mặt Trái Đất là $g=\SI{9.81}{\meter/\second^2}$, khối lượng của Mộc Tinh bằng 318 lần khối lượng của Trái Đất, đường kính của Mộc Tinh và của Trái Đất lần lượt là $\SI{142980}{\kilo\meter}$ và $\SI{12750}{\kilo\meter}$.
\begin{mcq}(4)
	\item $\SI{278.2}{\meter/\second^2}$.
	\item $\SI{24.8}{\meter/\second^2}$.
	\item $\SI{3.88}{\meter/\second^2}$.
	\item $\SI{6.2}{\meter/\second^2}$.
\end{mcq}
\hideall{
\textbf{Đáp án B.}\\
Gia tốc rơi tự do ở bề mặt hành tinh
$$g=\dfrac{GM}{R^2}$$
Như vậy:
$$\dfrac{g'}{g}=\dfrac{M_\text{Mộc Tinh}}{M_\text{TĐ}}\cdot\left(\dfrac{R_\text{TĐ}}{R_\text{Mộc Tinh}}\right)^2=\dfrac{M_\text{Mộc Tinh}}{M_\text{TĐ}}\cdot\left(\dfrac{D_\text{TĐ}}{D_\text{Mộc Tinh}}\right)^2=318\cdot\left(\dfrac{\SI{12750}{\kilo\meter}}{\SI{142980}{\kilo\meter}}\right)^2=2,53$$
$$\Rightarrow g'=\SI{24.8}{\meter/\second^2}.$$
}

\item Người ta phóng một con tàu vũ trụ từ Trái Đất bay về hướng Mặt Trăng. Biết rằng khoảng cách từ tâm Trái Đất đến tâm Mặt Trăng gấp 60 lần bán kính $R$ của Trái Đất, khối lượng của Mặt Trăng nhỏ hơn khối lượng Trái Đất 81 lần. Hỏi ở cách tâm Trái Đất là bao nhiêu thì lực hút của Trái Đất và của Mặt Trăng lên con tàu vũ trụ sẽ cân bằng nhau?
\begin{mcq}(4)
	\item $50R$.
	\item $60R$.
	\item $54R$.
	\item $6R$.
\end{mcq}
\hideall{
\textbf{Đáp án C.}\\
Gọi $R_1$ là khoảng cách từ tàu vũ trụ đến tâm Trái Đất và $R_2$ là khoảng cách từ tàu vũ trụ đến tâm Mặt Trăng.\\
Lực hút của Trái Đất và của Mặt Trăng lên con tàu vũ trụ cân bằng nhau thì
\begin{equation}
	\label{eq:1}
	G\dfrac{mM_\text{TĐ}}{R^2_1}=G\dfrac{mM_\text{MT}}{R^2_2}\Rightarrow \dfrac{R_1}{R_2}=\sqrt{\dfrac{M_\text{TĐ}}{M_\text{MT}}}=9
\end{equation}
Mà 
\begin{equation}
	\label{eq:2}
	R_1+R_2=60R
\end{equation}
Từ (\ref{eq:1}) và (\ref{eq:2}) ta thu được $R_1=54R$ và $R_2=6R$.
}

\item Một vệ tinh chuyển động tròn quanh tâm Trái Đất với bán kính quỹ đạo $\SI{6600}{\kilo\meter}$ với chu kì 89 phút. Biết hằng số hấp dẫn $G=\SI{6.67E-11}{\dfrac{\newton\meter^2}{\kilogram^2}}$. Khối lượng Trái Đất là
\begin{mcq}(4)
	\item $\SI{5E20}{\kilogram}$.
	\item $\SI{6E24}{\kilogram}$.
	\item $\SI{3E25}{\kilogram}$.
	\item $\SI{8E26}{\kilogram}$.
\end{mcq}
\hideall{
\textbf{Đáp án B.}\\
Vệ tinh chuyển động tròn xung quanh tâm Trái Đất, lực hấp dẫn do Trái Đất tác dụng lên vệ tinh đóng vai trò là lực hướng tâm
	$$F_\text{hd}=ma_\text{ht}$$
	$$\Leftrightarrow G\dfrac{mM}{R^2}=m\omega^2R=m\cdot\left(\dfrac{2\pi}{T}\right)^2R$$
	$$\Rightarrow M=\dfrac{1}{G}\cdot\left(\dfrac{2\pi}{T}\right)^2R^3\approx
	\SI{5.97E24}{\kilogram}$$

}

\item Trái Đất quay quanh Mặt Trời trên một quỹ đạo gần tròn có bán kính trung bình $\SI{150E6}{\kilo\meter}$. Biết khối lượng của Mặt Trời là $\SI{1.97E30}{\kilogram}$. Lấy $G=\SI{6.67E-11}{\dfrac{\newton\meter^2}{\kilogram^2}}$. Với các dữ kiện như trên thì chuyển động của Trái Đất quanh Mặt Trời là bao lâu?
\begin{mcq}(4)
	\item 365 ngày.
	\item 366 ngày.
	\item 367 ngày.
	\item 368 ngày.
\end{mcq}
\hideall{
\textbf{Đáp án D.}\\
Lực hấp dẫn do Mặt Trời tác dụng lên Trái Đất đóng vai trò là lực hướng tâm làm cho Trái Đất chuyển động trên quỹ đạo tròn xung quanh Mặt Trời
$$F_\text{hd}=ma_\text{ht}$$
$$\Leftrightarrow G\dfrac{mM_S}{R^2}=m\omega^2R=m\cdot\left(\dfrac{2\pi}{T}\right)^2R$$
$$\Rightarrow T=\sqrt{\dfrac{4\pi^2R^3}{GM_S}}\approx\SI{31.84E6}{\second}=368,56\ \text{ngày}.$$
}


\item Một vệ tinh chuyển động trên quỹ đạo tròn bán kính bằng nửa bán kính quỹ đạo của Mặt Trăng quay xung quanh Trái Đất. Biết chu kì quay của Mặt Trăng xung quanh Trái Đất là 27,5 ngày. Chu kì quay của vệ tinh nói trên xung quanh Trái Đất là bao lâu?
\begin{mcq}(4)
	\item 5,3 ngày.
	\item 6,4 ngày.
	\item 8,2 ngày.
	\item 9,7 ngày.
\end{mcq}
\hideall{
\textbf{Đáp án D.}\\
Vệ tinh chuyển động tròn xung quanh tâm Trái Đất, lực hấp dẫn do Trái Đất tác dụng lên vệ tinh đóng vai trò là lực hướng tâm
$$F_\text{hd}=ma_\text{ht}$$
$$\Leftrightarrow G\dfrac{mM}{R^2}=m\omega^2R=m\cdot\left(\dfrac{2\pi}{T}\right)^2R$$
$$\Rightarrow \dfrac{T^2}{R^3}=\dfrac{4\pi^2}{GM}=const$$
Như vậy:
$$\dfrac{T^2_\text{vệ tinh}}{R^3_\text{vệ tinh}}=\dfrac{T^2_\text{MT}}{R^3_\text{MT}}\Rightarrow T_\text{vệ tinh}=T_\text{MT}\cdot\left(\dfrac{R_\text{vệ tinh}}{R_\text{MT}}\right)^{\dfrac{3}{2}}=T_\text{MT}\left(\dfrac{1}{2}\right)^{\dfrac{3}{2}}\approx 9,7\ \text{ngày}.$$
}
\end{enumerate}