\chapter{Tóm tắt lý thuyết}
\section{Sóng và sự truyền sóng}
\begin{itemize}
	\item Sóng là dao động lan truyền trong không gian theo thời gian. Khi sóng truyền di, phần tử môi trường không truyền theo phương truyền sóng mà chỉ dao động tại chỗ.
	\item Dựa trên mối liên hệ giữa phương truyền sóng và phương dao động, sóng được phân thành hai loại:
	\begin{itemize}
		\item Sóng dọc là sóng mà phương dao động của mỗi phần tử môi trường trùng với phương truyền sóng.
		\item Sóng ngang là sóng mà phương dao động của mỗi phần tử môi trường vuông góc với phương truyền sóng.
	\end{itemize}
\item Các hiện tượng đặc trưng của sóng: phản xạ, khúc xạ, nhiễu xạ và giao thoa.
\end{itemize}
\section{Các đặc trưng vật lí của sóng}
\begin{itemize}
	\item Bước sóng $\lambda$ là quãng đường sóng truyền đi được trong một chu kì dao động $T$:
	$$\lambda=vT.$$
	\item Tốc độ truyền sóng là tốc độ lan truyền dao động trong không gian. Tốc độ truyền sóng trong không gian là hữu hạn và phụ thuộc vào tính chất của môi trường truyền sóng như mật độ môi trường, tính đàn hồi, nhiệt độ, áp suất, \dots\\
	Tốc độ lan truyền sóng cơ trong các môi trường: $v_\text{rắn}>v_\text{lỏng}>v_\text{khí}$.
	\item Cường độ sóng $I$ là năng lượng sóng truyền qua một đơn vị diện tích trong một đơn vị thời gian:
	$$I=\dfrac{E}{S\Delta t}=\dfrac{\calP}{S}.$$
\end{itemize}
\section{Sóng điện từ}
\begin{itemize}
	\item Sóng điện từ là điện từ trường lan truyền trong không gian:
	\begin{itemize}
		\item Sóng điện từ là sóng ngang.
		\item Sóng điện từ truyền trong chân không với tốc độ $c=\SI{3E8}{\meter/\second}$. Trong môi trường vật chất, tốc độ truyền của sóng điện từ đều nhỏ hơn $c$.
		\item Một số hiện tượng đặc trưng của sóng điện từ là: phản xạ, khúc xạ, nhiễu xạ, \dots
	\end{itemize}
\item Ánh sáng có bản chất là sóng điện từ. Bước sóng của ánh sáng có tần số $f$ trong chân không $\lambda=\dfrac{c}{f}$.
\item Thang sóng điện từ:
\end{itemize}
\begin{center}
	\begin{longtable}{|m{3em}|m{5.5em}|m{5.5em}|m{5.5em}|m{5.5em}|m{5.5em}|m{5.5em}|}
		\hline
		\thead{}& \thead{Ánh sáng\\ nhìn thấy} & \thead{Tia hồng\\ ngoại (IR)} & \thead{Tia tử ngoại\\ (UV)} & \thead{Sóng\\vô tuyến} & \thead{Tia X}& \thead{Tia gamma}\\
		\hline
		\textbf{Nhìn thấy bằng mắt thường}& Có & \multicolumn{5}{c|}{Không}\\
		\hline
		\textbf{Bước sóng} & $\SI{0.38}{\micro\meter}$ đến $\SI{0.76}{\micro\meter}$& $\SI{0.76}{\micro\meter}$ đến $\SI{1}{\milli\meter}$ & $\SI{10}{\nano\meter}$ đến $\SI{400}{\nano\meter}$ & $\SI{1}{\milli\meter}$ đến $\SI{100}{\kilo\meter}$ & $\SI{30}{\pico\meter}$ đến $\SI{3}{\nano\meter}$ & $\SI{E-5}{\nano\meter}$ đến $\SI{0.1}{\nano\meter}$\\
		\hline
		\textbf{Nguồn phát} & Mặt Trời, một số
		loại đèn, tia chóp, ngọn lửa, \dots &Vật có nhiệt độ cao hơn môi trường xung quanh thì phát được tia hồng ngoại ra môi trường. Nguồn thông dụng là bóng đèn dây tóc, bếp gas, bếp than, diode hồng ngoại, \dots & {Vật có nhiệt độ trên $\SI{2000}{\degree\celsius}$ thì phát ra tia tử ngoại, nhiệt độ của vật càng cao thì bước sóng tia tử ngoại phát ra càng nhỏ.
			
		Hồ quang điện, đèn hơi thuỷ ngân là nguồn phát tia tử ngoại mạnh.} & Được phát ra từ anten và được sử đụng để "mang" các thông tin như âm thanh, hình ảnh đi rất xa.
		& Được tạo ra khi các electron chuyển động với tốc độ cao tới đập vào tấm kim loại có nguyên tử lượng lớn trong ống tia X (ống Cu-lít-giơ) & Trên Trái Đất, tia gamma thường sinh ra bởi sự phân rã gamma từ đồng vị phóng xạ tự nhiên và bức xạ thứ cấp từ các tương tác với các hạt trong tia vũ trụ.\\
		\hline
		\textbf{Tính chất và ứng dụng} & Ánh sáng đỏ có bước sóng dài nhất $\SI{0.76}{\micro\meter}$ (tần số và năng lượng nhỏ nhất). Ánh sáng tím có bước sóng ngắn nhất $\SI{0.38}{\micro\meter}$ (tần số và năng lượng lớn nhất). & 
			- Tác dụng nổi bật là tác dụng nhiệt $\rightarrow$ sưởi ấm, sấy khô.
			
			- Chụp ảnh, quay phim ban đêm.
			
			- Có khả năng biến điệu như sóng điện từ cao tần $\rightarrow$ Điều khiển từ xa và truyền tin.
		& 
			- Tác dụng mạnh lên kính ảnh.
			
			- Kích thích nhiều phản ứng hoá học.
			
			- Ion hoá không khí.
			
			- Tác dụng sinh học: huỷ diệt tế bào $\rightarrow$ sát trùng, khử khuẩn.
			
			- Chữa bệnh còi xương.
			
			- Làm phát quang một số chất $\rightarrow$ phát hiện vết nứt nhỏ, vết xước trên bề mặt sản phẩm.
		&	
			- Sóng vô tuyến dùng trong truyền thanh, truyền hình được phát ra từ anten thì bị phản xạ bởi tầng điện li trước khi tới máy thu.
			
			- Sóng VHF (Very High Frequency) và sóng UHF (Ultra High Frequency) được sử dụng cho các đài phát thanh và truyền hình địa phương.
			
			- Sóng vi ba được sử dụng cho viễn thông quốc tế và chuyển tiếp hình ảnh qua vệ tinh thông tin và cho mạng điện thoại di động qua tháp vi ba.
	& 
		- Tính chất nổi bật của tia X là khả năng đâm xuyên mạnh $\rightarrow$ dùng phát hiện bọt khí bên trong sản phẩm, kiểm tra hành lý ở sân bay, \dots
		
		- Làm đen kính ảnh $\rightarrow$ dùng để chiếu điện, chụp điện.
		
		- Làm phát quang một số chất.
		
		- Ion hóa không khí.
		
		- Tác dụng huỷ diệt tế bào $\rightarrow$ dùng chữa ung thư nông.
&
	- Trong y học, tia garnma được dùng trong phẫu thuật, điều trị các căn bệnh liên quan đến khối u, dị dạng mạch máu, các bệnh chức năng của não.
	
	- Tia gamma còn được ứng dụng trong lĩnh vực công nghiệp. Tia gamma giúp phát hiện, các khuyết tật bằng hình ảnh rõ ràng với độ chính xác cao.
\\
\hline
	\end{longtable}
\end{center}
\section{Giao thoa sóng}
\begin{itemize}
	\item Hiện tượng giao thoa sóng là hiện tượng hai sóng kết hợp gặp nhau, tăng cường nhau hoặc làm suy yếu nhau tại một số vị trí trong môi trường.
	\item Hai sóng kết hợp là hai sóng thoả điều kiện:
	\begin{itemize}
		\item cùng phương dao động;
		\item cùng tần số;
		\item độ lệch pha không đổi theo thời gian.
	\end{itemize}
	\item Hiện tượng giao thoa ánh sáng là hiện tượng xuất hiện các vạch sáng xen kẽ các vạch tối khi hai sóng ánh sáng kết hợp gặp nhau.
	\item Khoảng vân là khoảng cách giữa hai vân sáng (hoặc hai vân tối) liên tiếp.
\end{itemize}
\section{Sóng dừng}
Sự giao thoa của hai sóng kết hợp truyền ngược chiều nhau trên cùng một phương tạo thành các bụng sóng (các điểm dao động với biên độ cực đại) xen kẽ với các nút sóng (các điểm đứng yên). Bụng sóng và nút sóng xen kẽ và cách đều nhau. Khoảng cách giữa hai nút sóng liên tiếp là $\dfrac{\lambda}{2}$.