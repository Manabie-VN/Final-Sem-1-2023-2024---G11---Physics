\chapter{Tóm tắt công thức}
\section{Các đại lượng đặc trưng của sóng}
\subsection{Chu kì, tần số}
$$T=\dfrac{\Delta t}{N};\quad f=\dfrac{N}{\Delta t}=\dfrac{1}{T}.$$
\subsection{Tốc độ truyền sóng}
$$v=\dfrac{s}{\Delta t}=\dfrac{\lambda}{T}=\lambda f.$$
\subsection{Cường độ sóng}
$$I=\dfrac{E}{S\dot \Delta t}=\dfrac{\calP}{S}.$$
\section{Phương trình sóng}
\subsection{Phương trình sóng}
Nguồn sóng O dao động theo phương vuông góc với trục $Ox$ với phương trình li độ:
$$u_\text{O}=A\cos\left(\omega t\right)$$
thì phương trình dao động tại điểm M cách nguồn O đoạn $x$:
$$u_\text{M}=A\cos\left(\omega t-\dfrac{2\pi x}{\lambda}\right).$$
\subsection{Độ lệch pha giữa hai điểm bất kì trên phương truyền sóng}
$$\Delta\varphi=\dfrac{2\pi d}{\lambda}$$
trong đó $d$ là khoảng cách giữa hai điểm trên phương truyền sóng.
\section{Giao thoa sóng}
\subsection{Giao thoa với 2 nguồn sóng cơ cùng pha}
Các điểm nằm trên đường trung trực của đoạn thẳng nối hai nguồn sẽ dao động với biên độ cực đại.
\begin{itemize}
	\item Vị trí các điểm dao động với biên độ cực đại thoả: $d_1-d_2=k\lambda, \quad k=0; \pm1; \pm2; \dots$.
	\item Vị trí các điểm dao động với biên độ cực tiểu thoả: $d_1-d_2=\left(k+0.5\right)\lambda, \quad k=0; \pm1; \pm2; \dots$.
\end{itemize}
\subsection{Giao thoa với 2 nguồn sóng cơ ngược pha}
Các điểm nằm trên đường trung trực của đoạn thẳng nối hai nguồn sẽ dao động với biên độ cực tiểu.
\begin{itemize}
	\item Vị trí các điểm dao động với biên độ cực đại thoả: $d_1-d_2=\left(k+0.5\right)\lambda, \quad k=0; \pm1; \pm2; \dots$.
	\item Vị trí các điểm dao động với biên độ cực tiểu thoả: $d_1-d_2=k\lambda, \quad k=0; \pm1; \pm2; \dots$.
\end{itemize}
\subsection{Giao thoa ánh sáng với thí nghiệm Young}
\begin{itemize}
	\item Khoảng vân: $$i=\dfrac{\lambda D}{a}$$
	trong đó:
	\begin{itemize}
		\item $i$: khoảng vân, đơn vị trong hệ SI là $\left(\si{\meter}\right)$;
		\item $\lambda$: bước sóng ánh sáng, đơn vị trong hệ SI là $\left(\si{\meter}\right)$;
		\item $D$: khoảng cách từ mặt phẳng chứa hai khe đến màn quan sát, đơn vị trong hệ SI là $\left(\si{\meter}\right)$;
		\item $a$: khoảng cách giữa hai khe hẹp, đơn vị trong hệ SI là $\left(\si{\meter}\right)$.
	\end{itemize}
\item Vị trí vân sáng: $x_s=ki,\quad k=0; \pm 1; \pm 2; \dots$
\item Vị trí các vân tối: $x_s=\left(k+0,5\right)i,\quad k=0; \pm 1; \pm 2; \dots$
\end{itemize}
\section{Sóng dừng}
\subsection{Vị trí bụng sóng/nút sóng}
Các bụng sóng và nút sóng các nút sóng cách đầu cố định
\begin{itemize}
	\item Vị trí bụng sóng: 
	$$d=\left(k+\dfrac{1}{2}\right)\dfrac{\lambda}{2}\quad\left(k=0; 1; 2; \dots\right).$$
	\item Vị trí nút sóng:
	$$d=k\dfrac{\lambda}{2}\quad\left(k=0; 1; 2; \dots\right).$$
\end{itemize}
\subsection{Điều kiện để có sóng dừng}
\subsubsection{Sóng dừng hai đầu cố định}
Chiều dài dây:
$$\ell=n\cdot\dfrac{\lambda}{2}=n\cdot\dfrac{v}{2f}, \quad n=1; 2; 3;\dots$$
Trên dây có:
\begin{itemize}
	\item $n$ bụng sóng;
	\item $n+1$ nút sóng.
\end{itemize}
Tần số của nguồn:
$$f=n\cdot\dfrac{v}{2\ell}$$
\begin{itemize}
	\item Hoạ âm cơ bản (bậc 1): $f_1=\dfrac{v}{2\ell}$.
	\item Hoạ âm bậc $n$: $f_n=nf_1$.
\end{itemize}
\subsubsection{Sóng dừng 1 đầu cố định và 1 đầu tự do}
Chiều dài dây:
$$\ell=\left(2n+1\right)\cdot\dfrac{\lambda}{4}=\left(2n+1\right)\cdot\dfrac{v}{4f}, \quad n=0; 1; 2;\dots$$
Trên dây có:
\begin{itemize}
	\item $n+1$ bụng sóng;
	\item $n+1$ nút sóng.
\end{itemize}
Tần số của nguồn:
$$f=\left(2n+1\right)\cdot\dfrac{v}{4\ell}=m\cdot\dfrac{v}{4\ell}$$
\begin{itemize}
	\item Hoạ âm cơ bản (bậc 1): $f_1=\dfrac{v}{4\ell}$.
	\item Hoạ âm bậc $m$: $f_m=mf_1, \quad m=1; 3; 5; \dots$.
\end{itemize}
